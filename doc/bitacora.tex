\documentclass[12pt,letterpaper]{article}
\usepackage[utf8]{inputenc}
\usepackage[T1]{fontenc}
\usepackage[activeacute,spanish]{babel}
\usepackage[left=18mm,right=18mm,top=21mm,bottom=21mm,letterpaper]{geometry}%
\usepackage{helvet}
\usepackage{amsmath,amsfonts,amssymb,commath}
\usepackage{graphicx}
\usepackage{color}
\usepackage{xcolor}
\usepackage{verbatim}
\usepackage{tabls}
\usepackage[space]{grffile}
\usepackage{url}
\usepackage{listings}
\usepackage{circuitikz}
\usepackage{siunitx}
\usepackage{fancyhdr}   
\pagestyle{fancy}
\usepackage{multicol,multirow}
\usepackage{textcomp}
\usepackage{booktabs}
\usepackage[colorlinks=true,urlcolor=blue,linkcolor=black,citecolor=black]{hyperref} 
\usepackage{pdfpages}   %incluir paginas de pdf externo, para los anexos
\usepackage{appendix}
\usepackage{caption}
\usepackage{subcaption}  
\usepackage{apacite}
\usepackage{natbib}
\usepackage{rotating}
\usepackage[section]{placeins}
\usepackage{tikz}
\lhead{ Laboratorio Microcontroladores}
\chead{}
\rhead{Propuesta: Laboratorio Especial}   % Aquí va el numero de experimento, al igual que en el titulo
\lfoot{Escuela de Ingeniería Eléctrica}
\cfoot{\thepage}
\rfoot{Universidad de Costa Rica}

%-------------------------------------------------------------

%\renewcommand{\labelenumi}{\alph{enumi}.}
%\addto\captionsspanish{\renewcommand{\tablename}{Tabla}}                    % Cambiar nombre a tablas
%\addto\captionsspanish{\renewcommand{\listtablename}{Índice de tablas}}     % Cambiar nombre a lista de tablas
\pagenumbering{Roman}
%------------------------------------------------------------

\author{ Ariel Fallas Pizarro, B42481\\ Daniel García Vaglio, B42781\\ Daniel Piedra Pérez, B45334
  \\ Esteban Zamora Alvarado, B47769 \\ 
\\ {\small Grupo 01}\\ Profesor: Enrique Coen \vspace*{3.0in}}


\title{Universidad de Costa Rica\\{\small Facultad de Ingeniería\\Escuela de Ingeniería
    Eléctrica\\SD HOST\\II ciclo 2016\\\vspace*{0.55in} Bitácora}}
%\date{12 de Octubre de 2016}

\begin{document}


\pdfbookmark[1]{Portada}{portada}

\maketitle
\thispagestyle{empty}
\newpage


\setcounter{page}{1}
\tableofcontents
%\newpage
%\listoffigures
%\listoftables

\newpage

\pagenumbering{arabic}


%--------------------------------
\section{Semana 1}
%-------------------------------

Reunión general del grupo para tomar la decisión de los blouqes que le corresponden a a cada
uno. Daniel García con DMA, Ariel Fallas con Register File, Daniel Piedra con CMD, Esteban Zamora
con DATA. 
\subsection{Ariel Fallas Pizarro}
\begin{itemize}
\item Lectura del Estándar para el SD Host
\end{itemize}
\subsection{Daniel García Vaglio}
\begin{itemize}
\item Lectura del Estándar para el SD Host
\end{itemize}

\subsection{Daniel Piedra Pérez}
\begin{itemize}
\item Lectura del Estándar para el SD Host
\end{itemize}

\subsection{Esteban Zamora Alvarado}
\begin{itemize}
\item Lectura del Estándar para el SD Host
\end{itemize}

\newpage
%------------------------------
\section{Semana 2}
%------------------------------

Reunión General para discutir el proceso completo de escritura y lectura de datos, y determinar lo
que se espera de cada bloque. 
\subsection{Ariel Fallas Pizarro}
\begin{itemize}
\item Definición de las entradas y salidas de los registros para los bloques del sd\_host.
\item Discusión con compañeros de otros grupos sobre como implementar los registros.
\item Separación de los distintos bloques a implementar.
\item Programación de registros.
\end{itemize}
\subsection{Daniel García Vaglio}
\begin{itemize}
\item Programación de la lógica de paso de estados para el DMA.
\item Establecimiento de Requerimientos para el funcionamiento del bloque
\item Programación del bloque de RAM, que se utiliza para simular a la computadora. 
\item Programación de un fifo simple, para hacer las pruebas iniciales del dma.
\end{itemize}


\subsection{Daniel Piedra Pérez}
\begin{itemize}
\item Establecimiento de los diferentes bloques necesarios para formar el bloque final
\item Asignación de las entradas y salidas de cada uno de los submódulos.
\item Asignación de los estados de las máquinas de estado de los bloques.
\end{itemize}

\subsection{Esteban Zamora Alvarado}
\begin{itemize}
\item Planteamiento de la arquitectura de los bloques (submódulos) para el apartado de control y
comunicación serial con la tarjeta SD.
\item Asignación preliminar de las entradas y salidas para los submódulos involucrados en el
  apartado de datos (DAT).
\item Lectura sobre sincronización de dominios de reloj distintos (CDC), así como sobre buffers
  asincrónicos para dos frecuencias de reloj.
\end{itemize}
\newpage
%--------------------------------
\section{Semana 3}
%-------------------------------
\subsection{Ariel Fallas Pizarro}
\begin{itemize}
\item Reajuste de los Registros.
\item Pruebas funcionales y de síntesis de Registros.
\item Programación del bloque de comunicación con el CPU.
\end{itemize}

\subsection{Daniel García Vaglio}
\begin{itemize}
\item Programación del sub bloque de fetch. Este se encarga de tomar el address descriptor de la
  memoria RAM. 
\item Programación del sub bloque de transfer. Este se encarga de realizar la transferencia de los
  datos entre memoria RAM y el buffer de datos. 
\item Se hacen pruebas de funcionamiento y se hacen pruebas de sintetización. Se verifica que no
  hayan Latches inferidos. 
\item Se termina de programar la máquina de estados principal. Pero no se realizan pruebas de
  síntesis.  
\end{itemize}

\subsection{Daniel Piedra Pérez}
\begin{itemize}
\item Programación del bloque CMD\_master.
\item Creación del testbench y probador del bloque CMD\_master.
\item Se verifica el comportamiento del bloque y se corrigen errores
\end{itemize}

\subsection{Esteban Zamora Alvarado}
\begin{itemize}
\item Planteamiento y programación de una versión inicial de las máquinas de estados para los
  submódulos de control y de comunicación serial con la tarjeta SD.
\item Escogencia e integración del buffer asincrónico de dos relojes en un módulo para la recepción
  y transmisión de datos, así como una serie de pruebas para mostrar su funcionamiento.
\item Programación de los estados de escritura de datos seriales hacia la tarjeta, a partir de los datos
  del buffer de transmisión. 
\end{itemize}
\newpage

%--------------------------------
\section{Semana 4}
%-------------------------------
\subsection{Ariel Fallas Pizarro}
\begin{itemize}
\item Pruebas finales para los bloques de registros y cpu\_communication.
\item Se habló con los compañeros para verificar que los registros a utilizar estuvieran funcionando correctamente.
\end{itemize}

\subsection{Daniel García Vaglio}
\begin{itemize}
\item Es necesario reescribir el código del sub bloque de transferencia de datos ya que no se teniaa
  contemplado el comportamiento cuando el buffer se llena o el buffer se vacía. Esto es muy
  importante ya que el dma funciona más rápido que el resto de los componentes y si los mensajes son
  muy largos entonces se tarda mucho ciclos en idle. 
\item Se verifica funcionamiento del nuevo bloque de transferencia, en conjunto con la máquina de
  estados anterior. 
\item Programación del dma. Este es un módulo que encapsula a la máquina de estados ya programada y
  tiene la lógica combinacional para adaptarse a los registros. 
\end{itemize}
\subsection{Daniel Piedra Pérez}
\begin{itemize}
\item Se programa el bloque el CMD\_physical.
\item Se realiza el testbench para el bloque CMD\_physical.
\item Se realiza el submodulo serial\_to\_parallel y el bloque parallel\_to\_serial.
\item Se realiza el testbench y probador de los bloques serial\_to\_parallel y parallel\_to\_serial.
\item Se corrigen errores con respecto a los bloques serial\_to\_parallel y parallel\_to\_serial.
\end{itemize}

\subsection{Esteban Zamora Alvarado}
\begin{itemize}
\item Programación de los estados de lectura de la tarjeta, para almacenar los datos en el buffer de recepción. 
\item Corrección de la estructura de la máquina de estados para considerar condiciones de full y
  empty en los buffers, así como la secuencia de CRC (espacio en la señal), inicio y final de cada bloque.
\item Implementación de las pruebas para verificar el funcionamiento adecuado de la transmisión de datos
\item Reordenamiento del código de los módulos en Verilog para corrección de errores de
  temporización detectados en las pruebas iniciales de los módulos implementados.
\end{itemize}
\newpage

%--------------------------------
\section{Semana 5}
%-------------------------------
\subsection{Ariel Fallas Pizarro}
\begin{itemize}
\item Conexión de los registros relacionados con la transmisión de datos con todos los demás bloques. 
\item Definición de los bits a los que puede o no puede escribir el CPU en el sd\_host.
\item Pruebas conjuntas simples de todos los bloques para transmisión de datos.
\item Realización de la presentación.
\item Ajustes de conexión.
\end{itemize}

\subsection{Daniel García Vaglio}
\begin{itemize}
\item Ajustes del dma para adaptarse al buffer real del proyecto, no es de pruebas internas.
\item Interconexión con el resto de bloques funcionales. 
\end{itemize}
\subsection{Daniel Piedra Pérez}
\begin{itemize}
\item Se realiza los ajustes necesarios para la correcta conexión con los otros bloques del SD Host.
\item Se realiza la presentación.

\end{itemize}

\subsection{Esteban Zamora Alvarado}
\begin{itemize}
\item Ajustes de los bloques de control y comunicación serial con la tarjeta SD para la integración
  con los demás módulos del SD Host, principalmente las banderas relacionadas con la lectura y
  escritura de registros.
\item Implementación de ciertos apartados de las pruebas de funcionamiento del SD Host relacionadas
  con la transmisión de datos seriales a la tarjeta SD.
\item Síntesis de los módulos implementados mediante Yosys, y correcciones leves a los mismos para
  lograr la equivalencia lógica con respecto a los módulos conductuales.
\item Se realizan ciertos apartados de la presentación del proyecto.
\end{itemize}
\newpage

\end{document}
